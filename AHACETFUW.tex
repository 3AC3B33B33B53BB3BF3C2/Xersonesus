%%%%%%%%%%%%%%%%%%%%%%%%%%%%%%%%%%%%%%%%%
% Dictionary
% LaTeX Template
% Version 1.0 (20/12/14)
%
% This template has been downloaded from:
% http://www.LaTeXTemplates.com
%
% Original author:
% Vel (vel@latextemplates.com) inspired by a template by Marc Lavaud
%
% License:
% CC BY-NC-SA 3.0 (http://creativecommons.org/licenses/by-nc-sa/3.0/)
%
%%%%%%%%%%%%%%%%%%%%%%%%%%%%%%%%%%%%%%%%%

%----------------------------------------------------------------------------------------
%	PACKAGES AND OTHER DOCUMENT CONFIGURATIONS
%----------------------------------------------------------------------------------------

\documentclass[10pt,a4paper,twoside]{article} % 10pt font size, A4 paper and two-sided margins

\usepackage[top=3.5cm,bottom=3.5cm,left=3.7cm,right=4.7cm,columnsep=30pt]{geometry} % Document margins and spacings

\usepackage[utf8]{inputenc} % Required for inputting international characters

\usepackage[T1]{fontenc} % Output font encoding for international characters

\usepackage{polyglossia}
\setdefaultlanguage{english}
\setotherlanguages{russian,ukrainian,chinese,turkish,arabic,french,italian,spanish,german,hebrew}
\setmainfont{Helvetica} % font book but only system fonts
\newfontfamily\cyrillicfont[Script=Cyrillic]{Helvetica}
\newfontfamily\hebrewfont[Script=Hebrew]{NewPeninimMT}
\newfontfamily\arabicfont[Script=Arabic]{Baghdad}



\usepackage{fontspec}

% \usepackage[T2A]{fontenc}      % Font encoding for Cyrillic
% \usepackage[utf8]{inputenc}     % Input encoding for UTF-8
% \usepackage[ukrainian]{babel}   % Ukrainian language support

\usepackage{times} % Use the Palatino font

\usepackage{microtype} % Improves spacings

\usepackage{multicol} % Required for splitting text into multiple columns

\usepackage[bf,sf,center]{titlesec} % Required for modifying section titles - bold, sans-serif, centered

\usepackage{fancyhdr} % Required for modifying headers and footers
\fancyhead[L]{\textsf{\rightmark}} % Top left header
\fancyhead[R]{\textsf{\leftmark}} % Top right header
\renewcommand{\headrulewidth}{1.4pt} % Rule under the header
\fancyfoot[C]{\textbf{\textsf{\thepage}}} % Bottom center footer
\renewcommand{\footrulewidth}{1.4pt} % Rule under the footer
\pagestyle{fancy} % Use the custom headers and footers throughout the document

\newcommand{\entry}[4]{\markboth{#1}{#1}\textbf{#1}\ {(#2)}\ \textit{#3}\ $\bullet$\ {#4}}  % Defines the command to print each word on the page, \markboth{}{} prints the first word on the page in the top left header and the last word in the top right

\usepackage{tikz}
\usetikzlibrary{calc} % Optional, but useful for more complex calculations

\usepackage{textcomp}

%\usepackage{hyperref}


%----------------------------------------------------------------------------------------

\begin{document}

%----------------------------------------------------------------------------------------
%	TITLE PAGE
%----------------------------------------------------------------------------------------


\begin{titlepage}

\AddToHookNext{shipout/background}{
    \begin{tikzpicture}[remember picture, overlay]
        \node at ($(current page.south) + (0, 8.25cm)$) % Adjust 1cm for desired distance from bottom
            {\centering\includegraphics[width=1.5\textwidth]{Regnum_Bosporanum}}; % Adjust width and image filename
    \end{tikzpicture}
}


  \centering % Center all content
    \vspace*{2cm} % Add vertical space from the top
    {\Huge Xersonesus \\} % Large title
    \vspace{1cm}
    {\Large An Historical and Critical Encyclopedia of the Ukraine War \\} % Subtitle
    \vspace{2cm}
    {\Large 3AC3B33B33B53BB3BF3C2 \\} % Author name
    {\normalsize \emph{ seppie coi piselli alla romana } \\}
    \vspace{1cm}
    {\today} % Date
    \vfill % Fill remaining vertical space
    % Add any other elements like logos, disclaimers, etc.


 \end{titlepage}

%----------------------------------------------------------------------------------------
%	SECTION A
%----------------------------------------------------------------------------------------

\section*{A}

\begin{multicols}{2}

\entry{ad hominem} {} {} {In the vast majority of cases where members of the community express opinions different from those of certain groups, representing hypotheses on the outcomes, solutions, or remedies for Ukraine's ultimate failure on the battlefield, the groups respond \textenglish{\emph{ad hominem}}.\footnote{You can read more about 𝑎𝑑 ℎ𝑜𝑚𝑖𝑛𝑒𝑚 arguments in Aristotle's 𝑆𝑜𝑝ℎ𝑖𝑠𝑡𝑖𝑐𝑎𝑙 𝑅𝑒𝑓𝑢𝑡𝑎𝑡𝑖𝑜𝑛𝑠, 178(b)17 where Aristotle describes the 𝑟𝑖𝑠𝑢𝑠 𝑠𝑜𝑝ℎ𝑖𝑠𝑡𝑖𝑐𝑢𝑠 in which 'persons direct their solutions against the man, not against his arguments.'} The groups attack the person but leave his or her arguments in tact. The most common theme for an attack \emph{ad hominem} is to claim man is the measure of all things\footnote{Protagoras, the pre-Socratic Greek philosopher (c.490–c.420 BC) espoused the view. He stated: "\textgreek{πάντων χρημάτων μέτρον ἐστὶν ἄνθρωπος, τῶν μὲν ὄντων ὡς ἔστιν, τῶν δὲ οὐκ ὄντων ὡς οὐκ ἔστιν}."} so that by a measure of expertise (however unqualified it may actually be) a man from the group is higher, stronger or smarter than the attacked. In the vast majority of these cases, the measure for height, strength or intelligence is hardly based upon anything one might associate with these concepts, especially in terms of analysis. The vast majority of the assailants can hardly muster a record on predictions commensurate with intelligence. } 

\entry{Artemovsk} {Donbas} {village} {See \textbf{Bakhmut-Artemovsk}.}

\entry{Artillery} {weapon} { } {In a phrase coined by the gravedigger of the revolution,\footnote{In his profound study of the social origins of the Soviet Union after Stalin's rise to power, \emph{The Revolution Betrayed}, Trotsky explained how without a worker revolution based on the theory of permanent revolution but aiming to reintroduce Lenin's democratic centralism into the Communist Party, the degenerated Soviet worker's state would inevitably collapse from stagnation. Stalin's decision to impose Socialism in One Country on the Communist Party together with a series of bloody reprisals designed to effectuate a political genocide against the generation of Octoberist workers confirmed Trotsky's analysis when the Soviet Union, suffering from years of Stalin's policies, disintegrated on 26 December 1991 by Declaration No. 142-N, a declaration the Soviet of the Republics of the Supreme Soviet of the Soviet Union made to dissolve the federation.} Joseph Stalin, artillery is the god of war. The god of war, however, is now but only a vital component in an evolving array of combined arms in eastern Europe. In the beginning of the Ukraine war the  media focused almost exclusively on parity. The number of Russian artillery guns, calibre, the number of munitions, the type of munitions, or the way in which the Russians accurately targeted Ukrainian targets demonstrated the advantage Russia weaned over its opponent. ّIn nearly every category, the Russians managed to demonstrate a noticeable superiority, a superiority not merely in quantity but in quality. The range of Russian calibres for artillery, for instance, far exceeded that of the Ukrainians. Confined to the 155mm M777 howitzers, the Ukrainians could not exercise fire control over battle areas in the way the Russians did. \newline \indent In the beginning of the Ukraine war, primarily before the great Ukrainian offensive in Kharkiv, launched on September 6th but largely until the end of Ukraine's 'Spring' counteroffensive in 2023, artillery played a dominant role on the battlefield. Ukraine's failed 'Spring' counteroffensive witnessed confirmed Russia's seizure of the strategic initiative after the fall of Bakhmut-Artemovsk on May 23rd, 2023. In the winter of 2023 the Ukraine war underwent a major transformation in the deployment of drones. Ukraine began to mass produce First Person View (FPV) drones to strengthen its defense of the Donbas in the face of continued Russian advances. Necessity is the mother of invention but the transformation could have come soon enough for Ukraine's defense of Avdiivka, a major turning point in the Ukraine war after the fall of Bakhmut-Artemovsk for Russian tactics (i.e., tunneling and FAB bombing with ”glide kits”) and Ukraine's deficiencies in artillery. Ukraine's FPV drones began to play a dominant role on the battlefield, overtaking but by no means substituting for the role artillery played in the beginning of the war. The first Ukrainian brigades with dedicated, trained, expert drone pilots began to take shape.}

\entry{Avdiivka} {Donbas} {village} {Ukraine's defense of Avdiivka, a major turning point in the Ukraine war after the fall of Bakhmut-Artemovsk for Russian tactics (i.e., tunneling and FAB bombing with ”glide kits”) and Ukraine's deficiencies in artillery and its deployment of FPV drones, collapsed on January 17th, 2024.}

\entry{Axe, David} {} {} {One of the more committed journalists of the Ukraine war, whose coverage of the Donbas, especially in the Avdiivka direction\footnote{One of the most important contributions to understanding the Ukraine's defeat in the Avdiivka direction is the 47th Mechanized Brigade in 2024. A prominent unit in the Ukrainian military, heavily involved in fighting Russian forces in the Avdiivka direction, particularly with Western-supplied equipment like M-1 Abrams tanks and M-2 Bradley fighting vehicles, the 47th Brigade perished within less than six months from its reconstitution. Deployed to Pavlohrad, 60 miles west of Pokrovsk, the front line in eastern Ukraine’s Donetsk Oblast, fought desperately with the 31 M-1 Abrams tanks to stop Russian advances. Consequently, the 47th Brigade lost more than half of its company of 31 M-1 Abrams tanks to Russian attacks. While the attacks claiming these tanks vary, the vast majority of the attacks stem from Russian FPV drones. The perished 47th Mechanized Brigade demonstrated that Western military equipment, though not obsolete by any means, was, nonetheless, no match for Russia on the battlefield in the Avdiivka direction. The destruction of the first M-1 Abrams tank, long considered to be the world's best tank, shattered the illusion of the overwhelming invincibility of American firepower. Not only were the tanks underwhelming, they were distinctively vulnerable.} from 2023 to the present, eclipsed others. Published primarily in \emph{Forbes}, these articles are an important source of information. At a time when the majority of news from the frontlines comes almost exclusively from 'milbloggers,' Axe's analyses shed light on rarely mentioned or hardly examined aspects of the Ukraine war. These aspects, though without the extensive detail of a comprehensive study, allow readers to imagine the whole of the subconscious iceberg from a partial view of its conscious tip. \newline \indent \emph{Forbes}, however, is a publication with a purpose. Its annual lists of America's four hundred richest Americans shows how its purpose, however, may converge with another. These lists, for instance, constituted one of the Socialist Equality Party's primary staples of propaganda in the years preceding the 2008 subprime mortgage crisis, the decade after the party's launch of the World Socialist Website in 1998\footnote{The SEP's launch preceded the Federal Bureau of Investigation's launch of a Federal Intrusion Detection Network, or FIDNET, modeled after a famous program from Linux's suite of open source security software called Snort, a network intrusion detection system (NIDS). Mentioned in the article published on  August 10th, 1999, under the title "White House plan for FBI Internet spying," FIDNET is one of the earliest precursors to the drag-net surveillance Edward Snowden exposed in his leak of secret documents from the National Security Agency (NSA). Originally exposed in an article published on July 28th, 1999, by the \emph{New York Times} under the title, "The U.S. Drawing Plan That Will Monitor Computer Systems," FIDNET is a part of the pre-history to the U.S. conspiracy to deprive citizens of life, liberty, or the pursuit of happiness through the technological obliteration of the founding father's Bill of Rights, especially the Fourth Amendment. Although the early SEP article makes no mention of the U.S. Constitution, the legal critique, so famous now, would only become more provocative as the U.S. conspiracy strengthened its execution.} under the pretense of a Marxian renaissance.\footnote{In an article published on October 31st, 1998 under the title, "International school examines the century’s central problems of history, politics and culture," the International Committee of the Fourth International emphasized how a "central premise guiding the school was that a renaissance of Marxism is fundamental to the development of a perspective that can answer the burning issues of the day—growing social inequality, deepening economic crises, the decline in the cultural level of society and the prevailing political paralysis of the workers movement."} The SEP's early articles\footnote{The first appears to be ["Gap between rich and poor is wider than ever," June 23, 1998] but many more follow: ["The Forbes 200 list: billions for the privileged few," June 30, 1999], ["Forbes 400 list of richest Americans: snapshot of a financial oligarchy," September 24th, 2004], ["Forbes reports bonanza for world’s billionaires," May 9th, 2005], ["The very rich in America: “The kind of money you cannot comprehend," April 19, 2006], ["Forbes publishes list of 400 richest Americans," October 6, 2006], ["The filthy rich: Forbes lists America’s top 400 for 2007," November 27th, 2007], ["America’s “Fortunate 400” control vast wealth," March 7th, 2008].} demonstrate this convergence. Opposed to the decade of doldrums, history's acceleration, however, has required expanded coverage of more fundamental political issues such as the Ukraine war. \newline \indent Axe's ability to achieve T.S. Eliot's ideal for the portrayal of an event, however tainted by his ambition for a Ukrainian victory, in his reporting on the war could not but be a source of concern for a magazine dedicated to celebrating the richest 400 Americans annually. The inversion of the invisible but highly detailed iceberg from the visible but only slightly detailed tip in his writing provided readers with far more with which to imagine the truth than receive a programmed vision of propaganda. It is simply unacceptable. It is pure blasphemy! It would thus only be a matter of time before \emph{Forbes}, whose aim is to articulate the interests of this wealthy elite, would clash with Axe's reporting on the Ukraine war, a war politicized to the extreme. At the end of the day, \emph{Forbes}, famous for its annual celebrations of the country's top one percent, decided to fire Axe, terminating his coverage before the end of the Ukraine war, the outcome of which has been to deprive not only the present but posterity of one its auspicious eyes.
}

%Ukraine pulls US-provided Abrams tanks from the front lines over Russian drone threats, AP, April 25th, 2024

\end{multicols}

%----------------------------------------------------------------------------------------
%	SECTION B
%----------------------------------------------------------------------------------------

\section*{B}

\begin{multicols}{2}

\entry{Bakhmut-Artemovsk} {} {} {The city whose fall sounded a death knell for the Ukrainian strategic initiative. Spelt \textukrainian{Бахмут — Артемівськ} in Ukrainian but —  \textrussian{Бахмут — Артемовск} in Russian, the city's history is unique. \newline \indent During the Nazi retreat in the aftermath of Operation Barbarossa's failure in 1943, Manstein, the 1st conquerer of Sevastopol, an expert in warfare against the Soviets, the Field Marshall whose counterstroke against the Soviets at Orel created the Kursk salient, organized Bakhmut-Artemovsk together Slavyansk and Kramatorsk into the Nazi lynchpin of defense for the Donbas. It is alleged that the Nazis withdrew before the Soviets managed to lay siege to the city, indicating that the Nazis never lost but forfeit the city. The fall of Bakhmut-Artemovsk therefore contained immense political, historical, military significance for the Russians. Ukraine's inability to stand the village up against the Russians terminated its ability to cling to the mythological power of its stand against the Soviets during the 1943 Nazi retreat, severing the early Ukrainian Galician narrative from Ukraine's armed forces. The Russians sacked the stronghold at Bakhmut-Artemovsk, preventing the Ukrainians from being able to revive the myth of Manstein's defense. The Ukrainian affinity for Nazism, renaming its streets after members of the Galician SS, stopped, dying silently without much of a trace in the media.

}

% \footnote{[“\textrussian{Крепкий напор и баталия. Как русские впервые отстояли Бахмут — Артемовск},” \textrussian{Регнум}, \textrussian{20 мая 2025}.]}

% \entry{Belarus} {} {} {

% Prigozhin
% Zapad-2025 in Belarus: Structure, Conduct, Implications. 

\entry{Black Hawk Helicopters} {} {} {
%https://notes.citeam.org/ru-dispatch-oct-31-nov-3-2025
}

\entry{Blackwater} {} {} {} 

\entry{Bombing} {bomb·ing} {Gerund} {Bombing has thus far been ineffective from the beginning of the Ukraine war. It alone has not accomplished a single strategic objective for any of the warring parties. \newline \indent The \emph{New York Times} is one of the most fervent advocates of bombing.\footnote{[“How to Choke Iraq,” \emph{NYT}, December 7th, 1990.]} \footnote{[“Bombing Iraq Ins’t Enough,” \emph{NYT}, January 30th, 1998.]} \footnote{[“American Bombs Make Iraq Stronger,” \emph{NYT}, December 20th, 1998.]} \footnote{[“Bomb North Korea, Before It’s Too Late,” \emph{NYT}, April 12th, 2013.]} \footnote{[“Bomb Syria, Even if It is Illegal,” \emph{NYT}, August 27th, 2013.]} \footnote{[“To Stop Iran’s Bomb, Bomb Iran,” \emph{NYT}, March 26th, 2015.]} With its well known history of advocating for bombing, the \emph{New York Times}' publication of an article condemning Russia's own bombing campaign during the latter end of 2022 contrasts sharply with its own history of opinion. 

}

\end{multicols}


%----------------------------------------------------------------------------------------
%	SECTION C
%----------------------------------------------------------------------------------------

\section*{C}

\begin{multicols}{2}

\entry{CIA} {Abbreviation} {Noun} {

It is often claimed in print that the CIA (Central Intelligence Agency) predicted the beginning of the Ukraine war. It is true. The CIA forecast Russia's full-scale invasion less than two weeks in advance. \footnote{ It is hard to imagine a time before the \emph{Kyiv Independent}, the newspaper the \emph{Wall Street Journal} tapped to cover the Ukraine war with the immediacy the U.S.-led NATO alliance demands for its rolling propaganda. \textukrainian{\emph{Українська правда}}, however, is one of the few Ukrainian newspapers whose history of publication precedes the struggle for Ukraine. It is thus a newspaper preceding the \emph{Kyiv Independent}'s coverage on the Ukraine war. It published the majority of the articles covering the CIA's predictions. In one of these articles, the headline describes the attack to be within days: \textukrainian{"Байден вважає, що Путін може напасти на Україну в найближчі дні} – The Guardian," \textukrainian{\emph{Українська правда}}, 11 \textukrainian{лютого} 2022. The first of these is reporting on \emph{The Guardian}. It is a reference to the \emph{The Guardian}'s article published on February 11th under the title, "US warns of ‘distinct possibility’ Russia will invade Ukraine within days." Reporting on \emph{Der Spiegel} revealed the date: ["\textukrainian{ЦРУ назвало імовірну дату нападу Росії на Україну} – \emph{Spiegel}," \textukrainian{\emph{Українська правда}}, 11 \textukrainian{Лютого} 2022]. A reference to the \emph{Der Spiegel}'s article published on February 11th under the title, "CIA rechnet mit russischem Angriff kommende Woche," the so-called \textukrainian{імовірна дата} could not have been vague in German. Another published a day after these read: ["\textukrainian{Головні новини п’ятниці та ночі: ймовірний напад Росії, санкції РНБО}, \textukrainian{\emph{Українська правда}}, 12 \textukrainian{Лютого} 2022"].} In the publications in Ukrainian, Hebrew, Russian, or English, the majority of the accompanying images are pulled directly from an article published seven years ahead of its time. On Mar 9, 2015, \emph{Stratfor} published its famous article, "Gaming a Russian Offensive." The scenarios the CIA discusses in its forecast are replications of those from \emph{Stratfor}'s article. The arrows in these sets of images follow those in \emph{Stratfor}'s article.  \newline \indent It is interesting how these new images spiral outward from a common source almost like a Fibonacci sequence, adding more arrows with the fractionalization of the original scenarios.  In one of the articles published by \textukrainian{\emph{Українська правда}}, the newspaper reports on February 12th how \textukrainian{"Росія має 9 маршрутів сэргнення до України, танки можуть сягнути Києва за 2 доби - розвідка США"}. A few of these nine routes, as impractical as the aforementioned fractionalization, could not have been farther from the truth, as the Russia massed scores of Russian troops on the highways into the Donbas, where the majority of the fighting raged for years before Russia launched its full-scale invasion. These three additional scenarios exceed \emph{Stratfor's} original six. 

}

\entry{Cocaine} {drug} {Noun} {Vladimir Putin, the President of the Russian Federation elected by more than 87.8 percent of the vote in Russia's last election in 2024 for a new six year term, becoming Russia's longest-serving leader for more than 200 years, famously declared his preference to work with those whose \textrussian{«Нос в кокаине!»} than \textrussian{«с умными»} in an interview with \textrussian{Димтрий Константинович Киселев} for \textrussian{«России 1»} and \textrussian{РИА Новости} on March 13th, 2024. The reference is in regard to negotiating an end to the Ukraine war. 
\newline \indent Putin claimed that intelligent politicians had by that time started to change strategies for relating to Russia, explaining that these people are dangerous because they want to throw \textrussian{«всякие свои хотелки под видом морковки»} (i.e., ulterior motives under the guise of proposals for a negotiated settlement).\footnote{ [ \textrussian{"Путин рассказал о предпочтении работать с теми, у кого «нос в кокаине»," Газета.ру,  13 марта 2024} ]. } It is therefore to be determined whether the start of serious negotiations for the end of the Ukraine war should start not with a peace proposal but with a mirror, razor or lines of cocaine, is it not? 

}

\entry{Construction Site} {} {} {Described by the \emph{Wall Street Journal} as "the world's largest open air construction site," Ukraine epitomizes Lenin's Chapter 5 and 6 division. } 


\end{multicols}


%----------------------------------------------------------------------------------------
%	SECTION D
%----------------------------------------------------------------------------------------

\section*{D}

\begin{multicols}{2}

\entry{} {} {} {
%DeepState
}

\entry{Delta} {Title} {Software} {Seldom mentioned in the media, Delta is a battlefield management system preceding the U.S. Army's phased released of its own Large Language Models for AI. Announced by Col. Jonathan Harvey from the U.S. Army on October 18th, 2025, the "sufficient, joint weapon target pairing Large Language Model (LLM)" is scheduled to run inside of the "kill chain," a model advancing from a simple exercise to a "fully devolved reasoning model." The most likely set of data is based on multiple projects running simultaneously in the field of artillery. One of these relates to completing the updates to NATO's firing tables, military artillery data sheets used for calculating trajectories, for the first time since the Cold War.\footnote{["NATO to make fresh push for common arms standards," \emph{Reuters} October 16th, 2024]} Another is likely the collocation of manuals on targeting, tactics, techniques or procedures\footnote{[https://www.cia.gov/readingroom/docs/CIA-RDP07S00452R000300820008-6.pdf]} for field artillery in combination with the aforementioned firing tables. In the face of the revolution the U.S. Army's phased release of its own Large Language Models is expected to achieve for combat management systems, Delta is now a legacy. 
\newline \indent In the first year of the Ukraine war, a Russian affiliated hacker by name of \textrussian{«Джокер»} managed to gain access to a Ukrainian commander's laptop with Delta on or around November 1st, 2022.\footnote{["\textrussian{Безсонов заявил, что хакер из ДНР взломал систему управления войсками ВСУ DELTA},"  \textukrainian{\emph{Известия}}, 1 \textrussian{ноября} 2022], ["\textrussian{Хакер Джокер рассказал подробности о взломе системы управления войсками ВСУ},"  \textukrainian{\emph{Известия}}, 1 \textrussian{ноября} 2022] } Described as a system for controlling Ukrainian troops, the system allegedly contains details about Ukrainian troops. At the time of the \textrussian{Джокер}'s announcement, Delta sounded like an unparalleled Ukrainian advantage in software development for the technology of warfare. In video footage from the article, however, Delta is an interactive map with real time updates. While the video footage is only one glimpse at the program, an interactive map could be hardly anything more than a 'novelty' by the New Year. A trend quickly emerged. The interactive map became a staple among journalists. The Ukrainian 'milblogger', DeepState, associated with Ukraine's intelligence services, championed updates to the "Map of the war in Ukraine," gaining recognition among the world's largest media companies. Military Summary, associated with Russia, is the first channel in the history of warfare to offer daily updates on the frontlines from the beginning of the Ukraine war. These 'mappers' overthrew the interactive maps with news, transforming war coverage forever. 

}


\entry{The Department of Defense} {Title} {Phrase} {In recognition of the role warfare plays in policy, the Trump administration reversed years of intellectually debased stagnation in the U.S. Armed Forces' bureaucracy with the elimination of the Department of Defense for the return of the Department of War during the Ukraine war. 

}

\entry{Donbas} {} {} {}


\end{multicols}

%----------------------------------------------------------------------------------------
%	SECTION E
%----------------------------------------------------------------------------------------

\section*{E}

\begin{multicols}{2}

\entry{Electricity} {} {} {  }

\end{multicols}

%----------------------------------------------------------------------------------------
%	SECTION F
%----------------------------------------------------------------------------------------

\section*{F}

\begin{multicols}{2}

\entry{FAB} {} {} {See "glide kits" (i.e., \textrussian{Унифицированный модуль планирования и коррекции (УМПК)})}.

\end{multicols}

%----------------------------------------------------------------------------------------
%	SECTION G
%----------------------------------------------------------------------------------------

\section*{G}

\begin{multicols}{2}

\entry{Gantz, David M. } {} {} {A detailed, thorough, insightful military historian of the eastern front, Gantz is one of the U.S. Army's untapped resources America's intelligence never managed to consult prior to the transformation of Julia Nuland's successful 2014 Maidan \emph{coup d'etat} into a full-scale provocation.\footnote{In an article published in \emph{Foreign Affairs}, John J. Mearsheimer, a veteran, under the title, "Why the Ukraine Crisis is the West's Fault," lays "most the of the responsibility for the crisis" on the United States and its European Allies. The taproot of the problem, Mearsheimer argues, "is NATO enlargement, the central element of a larger strategy to move Ukraine out of Russia's orbit and integrate it into the West." ["How the West Provoked Putin," \emph{Foreign Affairs},"See America: Land of Decay and Dysfunction," September/October, 2014] "For Putin, the illegal overthrow of Ukraine's democratically elected and pro-Russian president-which he rightly labeled a "coup"-was the final straw. He responded by taking Crimea, a peninsula he feared would host a NaTo naval base, and working to destabilize Ukraine until it abandoned its efforts to join the West."} Gantz's book, \emph{Operation Barbarossa: Hitler's Invasion of Russia 1941} provides a detailed historical account of the failed German offensive launched on June 22, 1941 against the Soviet Union. }

\entry{gold} {} {} {} {

In a historic rise unlike any previous period of time, gold's significance against the backdrop of the Ukraine war has emphasized the role currencies play in the struggle for the great powers to become the only superpower. On February 1st, 2022, a troy ounce of gold cost 1797.91	dollars. On October 17th, 2025, the price for a troy ounce of gold surpassed 4,378.69 dollars, the highest in the precious metal's history. Sanctions, more than any other factors, has played the greatest role in the rise of gold.  

}

\end{multicols}

%----------------------------------------------------------------------------------------
%	SECTION H
%----------------------------------------------------------------------------------------

\section*{H}

\begin{multicols}{2}

\entry{ } {} {} {}

\end{multicols}

%----------------------------------------------------------------------------------------
%	SECTION I
%----------------------------------------------------------------------------------------

\section*{I}

\begin{multicols}{2}

\entry{Institute for the Study of War} {} {} {The Institute for the Study of War (ISW) has consistently maintained that Ukraine may become victorious only after restoring maneuver to the battlefield. \newline \indent The ISW, one of the social media dedicated to downplaying Ukrainian losses, frequently disclaimed admitted Russian gains with immediate forecasts for their long-term insignificance in the war. Although the ISW has consistently downplayed Ukrainian losses, the reporting comes from fertile minds raised on the books from the ISW's library. Primarily the products of interns, the ISW's reports are neither from the military industrial complex nor the U.S. Army (which is the most intelligent of the four), neither the Central Intelligence Agency nor the network of contractors employed by the CIA in a covert fashion to act as independent commentators. These are primarily \emph{War on the Rocks}, the \emph{Rob Lee} group, the only secondary source the \emph{New York Times} quotes, or others such as those that may publish in \emph{Rand}, \emph{Foreign Affairs} or \emph{Foreign Policy}, etc... The ISW interns are neither affiliated with nor influenced primarily by MI6, Sandhurst (i.e., an excellent university with a great tradition in the study of war), or other English intelligence outfits. The ISW's reports thus represent a unique contribution to the study of the Ukraine war, diversifying the pro-Ukrainian pool of opinions with distinctly different points of view. \newline \indent The ISW interns typically receive welcomed support. Accordingly to posts on Twitter/X, they receive a stipend around 1,500 dollars, room, board, a library card to the ISW's library, a laptop, an office, courses, or opportunities to enhance their knowledge of warfare. 

}

\end{multicols}

%----------------------------------------------------------------------------------------
%	SECTION J
%----------------------------------------------------------------------------------------

\section*{J}

\begin{multicols}{2}

\entry{June 23rd, 2023} {} {} {In one of the greatest psychological operations in the history of the Western world, Vladimir Putin staged a \emph{coup d'etat} with Evgeny Prigozhin, the head of Russia's Private Military Corporation, Wagner, an elite special force staffed by experienced, well trained, career military men with a long history, most especially in the Middle East and Africa.}

\end{multicols}

%----------------------------------------------------------------------------------------
%	SECTION K
%----------------------------------------------------------------------------------------

\section*{K}

\begin{multicols}{2}

\entry{Kupiansk} {} {} {}

\entry{Kursk} {} {} {

%أميركا تطلب من أوكرانيا تسليمها السيطرة على خط ينقل الغاز الروسي لأوروبا


} 


\end{multicols}

%----------------------------------------------------------------------------------------
%	SECTION L
%----------------------------------------------------------------------------------------

\section*{L}

\begin{multicols}{2}

\entry{Light} {} {} {In one of the future developments of the Ukraine war, fiber optic drones, which are less susceptible to Directed Energy Weapons, High Powered Microwaves, or Electro-Magnetic Pulses, are expected to undergo a further transformation with the incorporation of light into the various components responsible for its flight. These components, mostly onboard electronics which are currently subject to disruption from a DEW, EMP or HPM, are expected to be designed with light. \newline \indent Light offers notable advantages not merely beyond immunity to these types of weapons. In terms of its ability to program realities in mathematics, light is capable of being configured to exhibit the properties theorized in group theory. For instance, passing light through a clear lens can serve as the unit; black lens as null; various other lenses as functions.

%Shahed 4

}

\entry{Lituania} {} {} {

%["La OTAN moviliza a dos cazas españoles tras la entrada de dos aviones rusos," \emph{El País}, October 23rd, 2025] ha anunciado una intrusión en su espacio aéreo de dos aviones rusos, lo que ha obligado a activar dos cazas Eurofighter españoles. “Nuestras fuerzas actuaron rápidamente con cazas de la OTAN”

} 

%\entry{Leonidas} {} {} {} 





\end{multicols}




%----------------------------------------------------------------------------------------
%	SECTION M
%----------------------------------------------------------------------------------------

\section*{M}

\begin{multicols}{2}

\entry{M1 Abrams} {} {} {
%https://notes.citeam.org/dispatch-apr-24-26-2024
}

\entry{Mad Max} {} {} { In an article published on April 6th, 2024 under the title, "Ukraine’s ‘Mad Max’ Trawls Swamps and Minefields for Shells," the \emph{Wall Street Journal} explained how Ukraine's ammunition shortage became so acute that an Ukrainian soldier, who hunts for Russian shells, became an important supplier for some units.}

\entry{Magura} {} {} {The highest expression of drone warfare on the open seas is the Ukrainian Magura. No weapon has wielded more power on the battlefield than the Magura, since no weapon has accomplished what the Magura has accomplished. In the aftermath of Bakhmut-Artemovsk, the Ukrainians lost complete control over the Black Sea. Shortly after losing control, the Ukrainians fought their way back by constantly iterating on the Magura drone, improving specific features until Russia lost more than half of its Black Sea Fleet, withdrawing the remainder to the port city of Novorossisk in Russia, far away from the Ukrainian Magura drones. The Magura drones represented one of the most sophisticated threats to sea vessels in the history of warfare. The Magura's ability to strike larger, more expensive, time-consuming constructions like ships such as corvettes, frigates, destroyers, etc... represents the essence of the drone threat, the ability to cause an enemy such irreplaceable loss that its restoration, several times the cost of the drone, takes more time than is feasible for its effect to impact the outcome of the war. 

}

\entry{Manstein, Erich von, the 2nd Conquerer of Sevastopol} {} {} {Since neither a Roman nor a n Ottoman commander captured the Cimmeria before Prince Grigory Aleksandrovich Potemkin-Tauricheski, Erich von Manstein, the Nazi general, became the 2nd Conquerer of Sevastopol after the city fell on July 4th, 1942. History, however, may call the general the first conquerer, since Manstein is the first to conquer the city, established by Catherine the Great, after its Russification into a naval fortress to assert her Empire's control over the Black Sea. 
\newline \indent Manstein conquered the Cimmeria with his personal dedication to one of the underlying principles of Blitzkrieg. He observed how the battle, which had raged for the better part of a year, began to tip the scales in the Nazi's favor.\footnote{During World War II, the U.S. Army's Military Intelligence Service initiated a series of reports on the Axis' powers. Among these reports is one dedicated to \emph{The German Armored Army}. The author expounds on the core tactical doctrine of German armor, explaining how "both from a tactical as well as from a strategical point of view, the selection of the maneuver to be carried out must always be inspired by the desire to disconcert the enemy command through its very boldness and rapidity. If necessary, 'what seems most improbable must be accomplished at the improbable place,' as remarked an officer of the staff of the 1st Armored Division in justification of the maneuver at Sedan." The author continues, stating: "By spectacular and even horrible combat methods, one must annihilate any will to resist on the part of the enemy. The fury of the "Stukas," the whizzing of their bombs, the din of their machine guns, the onrush of the tanks, the thunder of their march and of their fire, the spurt of the flames from the flame throwers, the explosion of melanin charges, everything must be brought into play to affect the morale of the combatant and give him the impression of an "apocalyptic" scene." Published on August 10th, 1942, Special Series \textnumero{ 2} is one of the U.S. Army's first explications of the Nazi's lightening warfare. What is fascinating about the term is just how flexibly its composition rendered absolutely new concepts to the German lexicon. In a depiction of the unbelievably expurgated cartography of Nazi campaigns from a book published in 1942 in \textgerman{Mǔnchen}, Germany, under the title, \textgerman{Der Krieg in Karten: 1939/1941}, the word \emph{Blitzsieg} comes to define the outcome of the battle over the Balkans from April 6th to the 30th, 1941 for the editor, Giselher Wirsing. Under the title, "\textgerman{Der Blitzsieg auf dem Balkan}" (pg. 24), an abstract depicts a lightening victory "\emph{aus dem Handgelenk heraus}." The German word victory, \emph{Sieg}, rhymes with war, \emph{Krieg}, indicating how the flexible nature of the German language accorded with dynamic concepts of warfare, easily creating new terms.} At precisely the moment when the scales began to weigh more heavily on the Nazis's side than the Soviets', Manstein commanded elements of his 11th Army to make the improbable possible with a cross-channel raid on Severnaya Bay where the Soviets estimated the chances for a successful attack to be highly improbable. In a reverberation of Manstein's sense for the appropriate place at the most opportune time like his modification of \emph{Fall Gelb}'s plan to pass through the Ardennes on May 10th, 1940, Manstein's correct appreciation for Napoleon's advice to maneuver according to circumstance prevailed.\footnote{It stands to mention that one of the unmentioned principles of \emph{Blitzkrieg} is harmony. In the Special Series \textnumero{ 2}, the author emphasizes how "it would be an error to compare one weapon with another. Far from encouraging rivalry among the various weapons, the new organization has developed their \underline{harmonious} association, and utilizes the motor to give them previously unrealized possibilities of speed. Armored weapons (light or heavy), infantry, artillery, engineers, signal communications, not to mention the air forces-all these arms contribute to the common aim: overcoming the adversary by an irresistible assault, followed by a complete destruction." \newline \indent Over the years, Sevastopol witnessed fewer conquerers than Jerusalem, the famous center of the world.}
}

\end{multicols}

%----------------------------------------------------------------------------------------
%	SECTION N
%----------------------------------------------------------------------------------------

\section*{N}

\begin{multicols}{2}

\entry{} {} {} { }

\entry{NATO's borders} {} {} { 

%https://www.alarabiya.net/alarabiya-today/2025/05/04/%D8%B1%D9%88%D8%B3%D9%8A%D8%A7-%D8%AA%D9%88%D8%A7%D8%B5%D9%84-%D8%AD%D8%B4%D8%AF-%D9%82%D9%88%D8%A7%D8%AA%D9%87%D8%A7-%D8%B9%D9%84%D9%89-%D8%AD%D8%AF%D9%88%D8%AF-%D8%A7%D9%84%D9%86%D8%A7%D8%AA%D9%88-%D9%88%D8%B2%D9%8A%D9%84%D9%8A%D9%86%D8%B3%D9%83%D9%8A-%D9%8A%D8%AD%D8%B0%D8%B1-%D9%85%D9%86-%D8%A7%D9%84%D9%85%D9%86%D8%A7%D9%88%D8%B1%D8%A7%D8%AA-%D8%A7%D9%84%D9%85%D8%B1%D9%8A%D8%A8%D8%A9-


}

\entry{Nigera} {} {} {Shortly after the fall of Bakhmut-Artemovsk (\emph{See} above), Nigeria became a focal point in the global context of the Ukraine war. Described as 'horizontal escalation', the head of Nigeria's government, which previously maintained a contract for stationing a U.S. base in the country, reneged, becoming one of the first countries to rebel against the rule's based order in Africa the United States could no longer guarantee with its presence alone. Subsequently, Nigeria joined neighboring states to form an alliance in the  Sahel region. Primarily designed to be for UAV reconnaissance, the U.S. air base accomplished little to nothing in the way of disrupting, eliminating or reducing terrorism in the area. The new government forced the U.S. to close the base before an embarrassing withdrawal. Months later, President Donald Trump issued a stark warning to Nigeria, threatening the country with invasion. \newline \indent Nigeria became only the first country in Africa to become the focal point in rapidly evolving struggle for spheres of influence among the great powers in Africa. 
}

\end{multicols}

%----------------------------------------------------------------------------------------
%	SECTION O
%----------------------------------------------------------------------------------------

\section*{O}

\begin{multicols}{2}

\entry{Oil} {} {} {%The highest expression of sanctions against Russia's oil industry became the sanctions on Russia's Rosneft and Lukoil companies. 

%\textgerman{Moskau reagiert auf Trumps Sanktionen gegen die Ölkonzerne Rosneft und Lukoil mit martialischen Durchhalteparolen – und mit Umgehungsstrategien. China und Indien zögern noch.\footnote{[ "Trumps neue Sanktionen: Russlands Ölindustrie unter Druck," \emph{Frankfurter Allgemeine}, October 23rd, 2025]}}


 }

\end{multicols}

%----------------------------------------------------------------------------------------
%	SECTION P
%----------------------------------------------------------------------------------------

\section*{P}

\begin{multicols}{2}

\entry{Pokrovsk} {} {} {A highly contested Ukrainian village where one of Ukraine's richest billionaires, the oligarch Rinat Akhmetov,\footnote{Described in the \textspanish{El País} as "\textspanish{El 'Abramovich' de los miners}," "\textspanish{el magnate del Shaktar}," Akhmetov has paid no single Ukrainian nor any member of his or her remaining family a death gratuity for the defense of his private assets in Pokrovsk or anywhere else in Ukraine. While private interests like Akhmetov's Metinvest Holding, LLC are the sole beneficiaries of Ukrainians' blood, sweat, and tears in the defense of Ukrainian villages, no slav has received a cent for spilling blood and guts on the Ukraine for oligarchs like Akhmetov, who has "\textspanish{la sexta mayor fortuna de los países del Este}. [September 28, 2004].} the owner of a Western aligned holding company called Metinvest Holding, LLC With operations in Ukraine, Italy, Bulgaria, the UK and the US, lost more than half of his total coal production in Ukraine on December 14th, 2024. \newline \indent It is possible to describe the battle of Pokrovsk in terms of a continuum stretching from the battle of Avdiivka to the city's fall on September 12th, 2025 when the \emph{Times} launched a calculated disinformation campaign designed to deceive readers into the false belief that Ukraine's defense of the city had succeeded in villages surrounding Pokrovsk. The news story, entitled, "They're encircled, and we'll wait until they surrender, withdraw or die' where the \emph{Times} claimed "Ukrainian troops surround hundreds of Russian infantrymen reviving hope after a brutal summer and, they say, swaying Trump in their favor." \footnote{The last part lasted no longer than a few weeks. Centered on the discussion of Tomahawk missiles, the missiles never came off the ground in the U.S. The Ukrainians are still waiting for those Tomahawks months later.} The article is one of the least well designed disinformation campaigns in the history of the war. The charts the author provides are transparently false. The diagram that it provides of Dobropillia does not display a circle; it does not demonstrate by way of shapes what is encircled. It has a series of fortified lines. These lines are not connected in a circle or encirclement. It show arrows from Ukrainian counterattacks. While an encirclement may ensue from a series of counterattacks, an encirclement arises after the exploitation of weak point leads to a line hardened by defenses. There are no lines hardened by defenses after or between the counterattack arrows. The fire brigade, Azov, already reconstituted more than four times with more than four changes in its commander, was already dispatched to Pokrovsk's defense but alas changed little to nothing for the Ukrainians dying to protect Metinvest Holding, LLC's assets for "\textspanish{El 'Abramovich' de los miners}." It was a last stage for securing the city from collapse but the stage fell through the floor. The characteristic lack of Ukrainian command was highlighted in the article when the Azov commander admits how only after being locked into a room for more than two weeks with a joint HQ, Corps command was restored. In the middle of a pitched battle lasting months, the fire brigade arrives to restore the chain of command. It reads like a foregone conclusion. 
}

\end{multicols}

%----------------------------------------------------------------------------------------
%	SECTION Q
%----------------------------------------------------------------------------------------

\section*{Q}

\begin{multicols}{2}

\entry{} {} {} { }

\end{multicols}


%----------------------------------------------------------------------------------------
%	SECTION R
%----------------------------------------------------------------------------------------

\section*{R}

\begin{multicols}{2}

\entry{rats} {} {} {Responsible for the malfunctioning French Caesar 155mm mobilized artillery, rats eat the delicious corn wiring.\footnote{In an article published in \emph{Le Figaro} on February 23rd, 2024, under the title, "\textfrench{Guerre en Ukraine: pourquoi les livraisons d’armes n’ont pas changé la donne}, the author explains the majority of the equipment the U.S.-led NATO alliance donated to Ukraine is not adapted to the Ukrainian battlefield, including the French Caesar 155 mobilized artillery. Quoting a Ukrainian soldier as saying, \textfrench{«Les rongeurs ont mangé les câbles sur certains véhicules»}, the author details rats on the eastern front immobilized the Caesar's engine. He writes: \textfrench{ «Les véhicules occidentaux ont été conçus comme une vitrine technologique. Mais dans la boue et le froid, ça ne fonctionne pas toujours».} The Russians, whose military equipment is rigorously tested, seized upon the French article, publishing an AI generated caricature of a single rat commanding a swarm of surrounding rats from atop a Caesar. [RT]}

}

\entry{railways} {} {} {Neither of the belligerents have achieved a distinct advantage from strategically bombing railways. }

\entry{Russia} {Russ·ia} {Noun} {Russia is one of the belligerents in the Ukraine war. In comparison to Ukraine, Russia's history of war in the area from the Black to the Baltic seas is extensive. 

}

\entry{raw materials, critical minerals or rare earths} {} {} {

%https://www.alarabiya.net/aswaq/special-stories/2025/05/04/%D8%A7%D9%83%D8%AA%D8%B4%D8%A7%D9%81-%D8%AB%D8%A7%D9%86%D9%8A-%D8%A3%D9%83%D8%A8%D8%B1-%D8%AD%D8%AC%D8%B1-%D8%A3%D9%84%D9%85%D8%A7%D8%B3-%D9%81%D9%8A-%D8%A7%D9%84%D8%B9%D8%A7%D9%84%D9%85-%D8%A8%D8%AF%D9%88%D9%84%D8%A9-%D8%A7%D9%81%D8%B1%D9%8A%D9%82%D9%8A%D8%A9

%\footnote{[“Pentagon steps up stockpiling of critical minerals with $1bn buying spree,” \emph{Financial Times}. October 11th, 2025]}

%\footnote{["The Countries Courting Trump with Critical Minerals," \emph{Foreign Policy}, October 23rd, 2025]}


}





\end{multicols}



%----------------------------------------------------------------------------------------
%	SECTION S
%----------------------------------------------------------------------------------------

\section*{S}

\begin{multicols}{2}

\entry{Sanctions} {Sanct·ions} {Noun} {

Sanctions, imposed on Russia after Russia launched its full-scale invasion of the Ukraine on February 24th, 2022, failed to produce the effect the U.S.-led NATO alliance sought. In the immediate aftermath of its imposition of sanctions, hundreds of thousands of Western businesses profiting from Russia's domestic market withdrew, causing billions of dollars in losses. Russia continued to sell oil. At the time, the price of oil skyrocketed. Russia earned no fewer than 93 billion Euros from profits on oil exports alone.\footnote{[\texthebrew{״סנקציות? רוסיה מכרה מאז הפלישה לאוקראינה נפט בשווי 93 מיליארד יורו. רק גרמניה קנתה נפט רוסי ב-12 מיליארד יורו.״}]}
 

}

\entry{Selydove, the fall of} {} {} {

}


\entry{Shahed} {Sha·hed} {Noun} {

Russia launched its first Shahed, one of the most important evolving weapons in the Ukraine war, in the immediate aftermath of its defeat in the face of the great Ukrainian offensive in Kharkiv   launched on September 6th, 2022. \newline \indent Debate rages around the modifications, versions, or models of Shaheds involved in legendary attacks, which Russian or Iranian factory produced them, the country of origin responsible for, or the actual nature, underlying engineering, or specification of the most distinguished parts, components or kits.  

}


\entry{'Spring' Counteroffensive} {} {} {The most peculiar aspect of Ukraine's 'Spring' counteroffensive, which well-known Ukrainian commanders call 'idiotic,' is just how well telegraphed its military objective became before its launch. The \emph{New York Times}, well before the Jack Texeira leaks, published article after article, detailing the 'Spring' counteroffensive's military objective, telegraphing to the Russians not only the upcoming offensive but its aim. In an article published in the \emph{New York Times} on March 8th, 2023, for instance, where the authors claim “Bakhmut is [the] Mercenary Group’s ‘Last Stand” in Ukraine, the authors say the “campaign will likely focus on the southern region of Zaporizhzhia, where Ukraine is building up forces, [as] Col. Roman Kostenko, a member of Ukraine’s Parliament who is serving in the country’s military, told Ukrainian television on Monday.” In another article published on March 13th, 2023 by the \emph{New York Times} under the title, “Kyiv Seeking to Evacuate a Town It Controls,” the logic, the authors write, is to capture “territory around Melitopol,” so that “Ukrainian forces [may] sever a Russian line of control from the Crimean Peninsula to the Donbas region.” On March 14th, 2023, \emph{the New York Times} in an article entitled, “Russian Attacks Yield Little but Casualties in Wide Arc of Ukraine’s East,” repeats: “the Ukrainians, anticipating a big influx of Western weaponry and fresh troops in the coming weeks and months, are widely expected to mount a counteroffensive." The article continues: "Analysts, Ukrainian officials, and even Russian commenters have suggested that it would come on the southwestern part of the front, with the Ukrainians [attempting] a push east from Kherson and south from Zaporizhzhia toward the city of Melitopol, hoping to sever the land bridge the Russians have seized that links the Crimean Peninsula to the eastern Donbas region.” Ukraine’s singular goal, we are told in an another article published on March 21st, 2023 with the title, “Little Time on the Battlefield to Dwell on Notions of Peace Talks,” is to bide time until Kyiv’s troops retake the initiative in the war. The “moment,” the authors explain, “remains unknown outside the close circle of the Ukrainian military high command.” It would seem that within the span of less than two weeks in March, 2023 the \emph{New York Times} exposed the Ukraine's  battle plans in their entirety well outside the close circle of Ukrainian military high command. Given the amount of time from March to June, four months, the \emph{New York Times} likely accomplished more for the exposition of Ukraine's battle plans than Russian intelligence itself. \newline \indent Finally launched in June, the 'Spring' counteroffensive started in summer. It lasted until November 2023 when multiple Armed Forces of Ukraine (AFU) brigades failed to penetrate the Russian Surovikin line along the Orikhiv-Tokmak Axis in Zaporizhzhia Oblast. Although the Ukrainian brigades advanced approximately 20km at the cost of 518 vehicles, including 91 tanks and 24 engineering vehicles, the Ukrainians never managed to breech Russia's first line of defense on the Surovikin line.\footnote{[“On One Key Eastern Battlefield, The Russians Are Losing 14 Vehicles For Every One The Ukrainians Lose,” Forbes, 14 November 2023]}

}

\entry{Stones} {} {} {At the height of Ukraine's crisis in artillery, a German speaking commentator, who sympathized with Ukraine's lack of munitions, declared how the Ukrainians would sooner start throwing stones than fire artillery shells at the rate the U.S.-led NATO alliance insisted European and Western nations contribute to the country's stocks.  Marc Thys, \textgerman{pensionierter Generalleutnant und früher der stellvertretende Chef der belgischen Armee}, made the statement in an interview to Bayerischer Rundfunk (BR) on February 3rd, 2024, just a few weeks before the third anniversary of the Ukraine war on February 24, 2024. \newline \indent Coming on the heel of the fall of Avdiivka, one of Russia's long-sought targets in Ukraine's collapsing manifest of a defense belt in the Donbas, on January 17th, 2024, the commentators describe Ukraine's munitions not in terms of numbers but time. Published on February 29th, 2024, five days after the third anniversary of the Ukraine war, the video Ukraine-Krieg: Gehen dem Westen die Waffen aus? Describes how "\textgerman{Die ukrainischen Truppen verschiessen jeden Tag Unmengen a Munition an der Front. Deutschland hat jetzt erst ein Flugabwehrsystem vom Typ geliefert und Artilleriemunition: 10,000 Schuss. Das reicht für zwei Tage. Denn die Ukraine bräuchte zwei bis 2, 4 Millionen Artilleriegranaten - pro Jahr. Das sind Schätzungen}." These measurements, which are based on Ukraine's rate of fire for a particular weapon system, symbolize how artillery is not a matter of stones, throwing, or throwing stones but time. 

% \href{https://www.youtube.com/watch?v=8MQdW00Jc4M}{Ukraine-Krieg: Gehen dem Westen die Waffen aus?}

}

\entry{Sudan} {} {} {An unrelated war, Sudan's second civil war began shortly before the fall of Bakhmut-Artemovsk on April 15th, 2023. Unlike Syria's civil war or the Sahel region, Sudan's second civil war is one of few processes in the struggle among great powers for spheres of influence in Africa to converge with any of the processes at stake in the Ukraine war's continuation in any direct way. It has remained for better or worse a primarily isolated conflict in northeastern Africa. Bordered by Egypt to the north, the Red Sea to the northeast, Eritrea and Ethiopia to the east, South Sudan to the south, and Chad, Libya, and the Central African Republic to the west, Sudan could easily become entangled in the sprawling network of African conflicts. Situated at the junction of the Blue Nile and White Nile rivers, the capital, Khartoum, is at one of the most important river ways in Africa. 

}



\entry{Surovikin, Sergey} {} {} {Sergey Surovikin, the Commander-in-Chief of the Russian Aerospace Forces from 2017 until he was reportedly sacked by Vladimir Putin, oversaw the battle of Bakhmut-Artemovsk from the withdrawal of Russia's armed forces from Kherson to the east bank of the Dnipro to the construction of the most famous Surovikin line, stretching from Belarus, a contractor for Russia's defense industry specialized in repair, maintenance, and restoration\footnote{[\texthebrew{המערך הצבאי של בלו רוסיה הפך לעורף הלוגיסטי של הצבא הרוסי הנלחם באוקראינה. משפץ מטוסים, טנקים, מערכות כ״מ ומכ״ם שנפגעו במלחה. תיק דבקה, אוג 22, 2022}]}, to the Dnipro delta. Appointed to the overall command of the Donbas theater immediately after Ukraine's great 2022 Kharkiv counteroffensive launched on September 6th, 2022, Surovikin began construction of the complex network of defenses immediately. In its concluding remarks on an analysis of the Surovikin lines, the Center for Strategic and International Studies stated: "Russia has constructed some of the most extensive systems of military defensive works seen anywhere in the world for many decades."\footnote{["Ukraine’s Offensive Operations: Shifting the Offense-Defense Balance," \emph{CSIS}, June 2nd, 2023.]} Composed of layers of defenses such as minefields, dragon's teeth\footnote{[\textarabic{أسنان التنين تكتيك روسي "قديم" لصد الهجوم الأوكراني المضاد،العربية،  : ٢٩ يوليو ٢٠٢٣}]}, anti-tank ditches, trenches, pill boxes, and other barriers designed specifically to entrap the incoming Ukrainian 'Spring' counteroffensive, the construction of the Surovikin lines relayed heavily a Russian advantage in equipment. The Russian BTM-3, a high speed, tracked, trench-digging vehicle based on a heavy artillery prime-mover chassis, digs trenches several feet deep. It is capable of digging 800 meters of man-sized trench in one hour in 3rd gear. The minimum radius is 25 meters. The Russian BTM-3, whose implementation in the construction of the Surovikin line exceeds all other equipment, is unparalleled; few, if any other nations, have vehicles such as these. 
\newline \indent A Russified response to warfare on the eastern front, the Surovikin line recalls the networks of fortifications the Russians utilized throughout the more than three centuries of warfare preceding the Ukraine war for protecting the road to Moscow. A series of ten  redoubts, defensive earthworks, designed to trap the Swedes in marshes during the battle of Pultova in 1709 is the first. The great redoubt at Borodino constructed under the auspices of one of Russia's most colorful generals, Mikhail Kutuzov, whose famous ability to defeat enemies, especially the Turks, in retreat cemented his preeminence in annals of Western warfare, upset Napoleon Bonaparte's bid to compel the Russians to capitulate in 1812. The earthworks one of Leon Trotsky's subordinates, Georgy Zhukov\footnote{\emph{The Great Commanders} (2003), an award winning expert portrait of six men, places Zhukov after Alexander the Great, Julius Caesar, Horatio Nelson, Napoleon Bonaparte, and Ulysses S Grant.}, one of the greatest Russian and Soviet military leaders of all time, built to prepare the Russian victory against the Nazis at Kursk on July 5th, 1943. Surovikin's construction of a major defensive network of earthworks thus places his generalship firmly and thoroughly within the line of the great Russian generals of the 18th, 19th, and 20th century. It is no wonder that Vladimir Putin sacked Surovikin, for Surovikin's preeminence unquestionably rivals all others in Russia right now, including Sergei Shoigu, Valery Gerasimov, and any other member of the Russian \textrussian{ставка}, not to mention Russian politicians. 

}

\entry{Switchblade} {} {} {The American Switchblade-300, Dead-on-Arrival on the Ukrainian battlefield, is an example of the Western belief in its superior firepower. Essential kit for America's special forces in Iraq and Afghanistan, the Middle East, the Switchblade-300 demonstrated its obsoletion in a way reflective of the times. As opposed to other types of weapons, a drone's effect is in its destruction; hence a drone's ability to disrupt, damage, or destroy a target must outweigh not only this but its own cost by an order of magnitude many times greater. The Switchblade-300, whose production facilities are overseas, costs many times more than an average FPV drone but without an immunity to Russia's electronic warfare. In an article published on October 23rd, 2025, Viktor Dolgopiatov, who heads Burevii, a design bureau pioneering [an] emergent class of [weapons],the "average ground drone (UGV) in Ukraine, for example, has a life expectancy of just one week."\footnote{["Western drones are underwhelming on the Ukrainian battlefield," \emph{The Economist}]} With a lifespan far shorter than expected for the Switchblade-300, the weapon suddenly became obsolete. \newline \indent The rapid evolution of drone warfare prompted the U.S. Army to take note of its obsoletion. In February 2022, the U.S. Army issued its first notice, requesting sources to supply Switchblade 300s, perhaps indicating a desire to strike a balance in a cost benefit ratio. In November 2022 the Army issued its second notice, submitting a Request for Information (RFI) on loitering munitions, directly referencing the conflict in Ukraine, which “has clearly demonstrated the ability of unmanned systems at increasingly lower echelons of employment,” especially for “[delivering] lethal effects.” \newline \indent In contrast with the Shahed, whose versions, modifications or models are studied in detail by both sides, the United States has not developed a loitering munition with a degree of intensity comparable to the Russo-Iranian project's emphasis on iteration. 

}



\entry{Syria} {Syr·ia} {Noun} {
The collapse of Syria's Assad regime, a fifty year old dynasty, is a result of the Ukraine war. Unique among the results of horizontal escalation, the collapse of Syria's Assad regime is a coalescence of processes arising from the Arab Spring with those of the Ukraine war. \newline \indent Al-Sharaa, an extremely clever Arab, trained a special division of his troops in drone warfare shortly before he launched his lightening offensive from Aleppo to Damascus, indicating how an element of the Ukraine war (ultimately derived from the Azerbaijanians in the \emph{Nagorno-Karabakh} war over Armenia\footnote{} complimented others such as the withdrawal of Russia's Private Military Company, Wagner, from Syria.\footnote{[\texthebrew{רוסיה מסיגה את כוחותיה בסוריה וריכזה אותם בשלושה שדות תעופה כהכנה להעביר חלק מהם לאוקראינה. מעבירה בסיסים למשמרות המהפכה וחיזבאללה}]}, both a result of Russia's full-scale invasion. While coverage on Al-Sharaa's lightening offensive is scant, if not brief, other elements receive a great deal of attention in the newspapers. In the mainstream Israeli press, the newspaper, \emph{Maariv}, published a series of articles detailing how nearly twelve months before the collapse of Syria's Assad regime Iran began to reduce its presence in the country.%\footnote{\texthebrew{[חוששת מישראל? איראן מצמצת את נוכחות בכירי משמרות המהפכה בסוריה , מעריב, 01.02.2024]}, \texthebrew{[איראן חוששת מישראל וצמצמה נוכחות בסוריה, בכירים במשמרות המהפכה עזבו ,ידיעות האחרונות, 01.02.2024]}, \texthebrew{[בעקבות גל החיסולים: "איראן מצמצמת את הנוכחות שלה בסוריה" ,01.02.2024]}. This element, undoubtedly a result of Israel's decision to expand its war with \emph{Palestina} from the Gaza Strip to Lebanon, is less related to the Ukraine war than the other elements. It clearly illustrates, however, the Grand Chessboard. For every action, there is an equal and opposite reaction in the zero sum cosmos. Alongside the fall of Bakhmut-Artemovsk, the collapse of the Assad regime represented one of the greatest moments in the geopolitics of the war.}



%\footnote{\texthebrew{[הפרשן הטורקי הימם את הסעודים: המטרה של ישראל היא לא רק להחליש את איראן יש עוד יעד, מעריב, 04.12.2024]





}


\end{multicols}


%----------------------------------------------------------------------------------------
%	SECTION T
%----------------------------------------------------------------------------------------

\section*{T}

\begin{multicols}{2}

\entry{to temper} {} {} {
In his famous book, \emph{The Prince}, Machiavelli asked on which side of a scale containing one of two animals the challenge for a ruler ruling over people weighs most heavily according to a given situation. On one side, Machiavelli sets the fox. On the other side, the lion. The weaker of the two is supposed to represent \emph{cunning} without strength. The stronger of the two \emph{brute force} with intelligence.  According to Machiavelli, a prudent ruler must be adaptive, changing to new situations, as problems arise, acting as a fox "in order to recognize traps" and as a lion when he has no choice but to "frighten off wolves." Machiavelli argues that the lion, the one symbolizing strength over intelligence, "is defenseless against traps" while the fox, the one symbolizing cunning over brute force, "is defenseless against wolves" and other physical threats; an effective ruler draws on the necessary, though not mutually exclusive, attributes of these great beasts, to keep his crown.
\newline \indent It would not be fair to dismiss Machiavelli's analogy outright for imprecision but no analogy escapes reality unscathed; Machiavelli's is no exception. What is a trap? Who lays traps? Who benefits from traps? Is there not room for a hunter? \footnote{In his unparalleled contribution to the study of being in the \emph{Metaphysics}, Aristotle, for instance, criticized Socrates' theory of the forms by virtue of the third man; if man is man by virtue of his participation in the form of man, a second form for participation is required. A third form relating the form for participation to that of the participant is thus required. The process continues ad infinitum. Socrates' theory of the forms thus becomes a wind egg. (c.f., \emph{Theaetetus} 161a) Is there no third man argument by which great philosophers like the Jewish monotheist, Aristotle, might make to undermine analogies like Machiavelli's? Is it not plausible to imagine a hunter behind the scenes of the fox and the lion? } \newline \indent It is clear that Machiavelli's understanding of the scales of beastliness or animality is nothing but to be contained in the eye of the \emph{arbiter elegantiarum}. An example is an elegant debate. A trap laid for the public, the two jaws of the foot piece, designed to catch the minds of readers, came in the form of a measure designed according to its counter-measure, the two corresponding momentanae.  \newline \indent The article published on April 12, 2024 by the \emph{New York Times} under the title "The Math on Ukraine Doesn't Add Up"  (the measure) was written so as to be a measure for its pre-existing counter-measure. Senator Roger Wicker's memo entitled, "Production is Deterrence," with a publication date on April 22nd, 2024 is the pre-existing counter-measure. With the measure before its counter-measure appearing sequentially, the Republican Senators, J. D. Vance and Roger Wicker, successfully tempered a trap for swaying American public opinion about "the math" on a range of issues related to Ukraine's war \textfrench{matériel} away from disapproval to support for funding. The math, which adds up in Senator Roger Wicker's memo, doesn't in J. D. Vance's article but publishing the latter before the former allows for latter to appear as a refutation. The refutation, however, was designed according to what was to be refuted well ahead of time. Senator Roger Wicker thus tempered V. D. Vance to publish his article before his memo.  \newline \indent Since the American public is never informed about the backroom discussions centering on or the creation, revision, or conversations with newspapers about, these documents, the sequential nature of forming the measure according to a pre-existing counter-measure is entirely concealed from the public's eyes, just like a hunter's trap is concealed from a fox's or lion's feet. Out of sight, out of mind. Machiavelli's analogy is thus a great starting point for identifying the humanistic aspects of governance. Why would any one have any need for speechless beasts when so much is clear from speech alone? \newline \indent  Passage of the funding sailed through the two houses of congress onto President Biden's lap in no time.\footnote{["US House approves 61bn in military aid for Ukraine after months of stalling," \emph{The Guardian}, April 20th, 2024]; ["What the US Congress passed to aid Ukraine, Israel and Taiwan," \emph{Reuters}, April 23rd, 2024]; ["US Congress passes Ukraine aid after months of delay," \emph{Reuters}, April 23rd, 2024] } While the Republicans passed funding for Ukraine's war \textfrench{matériel}, funding made no apparent difference, as the Ukraine failed to change the status quo ante on the battlefield in any noticeable way. None of the Ukrainian newspapers, such as the \emph{Kyiv Independent} or \textukrainian{\emph{Українська правда}}, published articles, describing how the newly passed aid for Ukraine or its new war \textfrench{matériel} made much of a difference. \newline \indent The problem is neither the hunter nor the trap but the fact that for the most part the animals never really realize what being trapped truly means. The failure to realize one is trapped is related almost exclusively to the fact that trap's objective is less to trap than to reinforce entrapment. The design of opposites, the counter-measure for the measure, ensnares people into thinking that these opposed points of view are really an opposition. The fact of the matter is that beyond the symmetrically designed teeth on opposite ends of a trap is a far greater wonder to behold. It is logic. 
}



% روسيا تتهم أوكرانيا بمهاجمة محطة لخط "ترك ستريم" للغاز

\entry{Time} {Ti·me} {Noun} {

It is to be determined which of the most prized possessions in warfare, time or space, the belligerents seek or exploit at any given phase of the Ukraine war. Neither, both of which most analysts measure by clock or ruler, are a matter of minutes, seconds, hours, days, weeks, months, or years or inches, feet, yards, or miles but of a more profound concept as germane to war as weapons. 

}


%«تجميد»


\entry{tunneling} {} {} {During the battle for Avdiivka, the Russians tapped the so-called Tiger\footnote{[\textarabic{"«النمر» قائداً للقوات الخاصة في سوريا... «تجميد» أم سباق نفوذ روسي - إيراني؟.  الشرق الأوسط. ١١ أبريل ٢٠٢٤"}]}, a Syrian warrior of partially Russian, partially Syrian origin, for inspiration. The Tiger, whose whereabouts immediately following the fall of the Assad regime spare no mention, save a few in February and March, 2025,\footnote{One post portrays the general leading troops in Syria's Alawite territories in March after reports of massacres in February. [\textarabic{سهيل الحسن يشعل الساحل السوري... ما نعرفه عن «النمر» مبتكر البراميل المتفجرة.  الشرق الأوسط.٩ مارس ٢٠٢٥"}] No subsequent mention follows.}, conceived of the barrel bombs on Aleppo during the Syrian civil war. The result became Russia's variously sized FAB bombs. It is during the battle for Avdiivka that Russians began to utilize these bombs for crushing Ukrainian defenses embedded deep within the fortress cities of the Donbas. These FAB bombs, however, were not enough to break the will of the Ukrainian defenders. \newline \indent Ultimately the Russians conceived of an old World War I tactical called tunneling. The Russians invaded the city's center from dry tunnels well beneath Avdiivka. It was the first time but certainly not the last time the Russians utilized the tactic during the Ukraine war. Published in a tweet on March 4th, 2024, RTArabic described a Russian soldier as saying, "The enemy expected us above but we came out from under ground."\footnote{\textarabic{["العدو كان ينتظرنا فوق الأرض لكننا خرجنا من تحتها قائد مجموعة اقتحام أفدييفكا يروي لفريق كيفية اقتحام المدينة وهروب القوات الأوكرانية."]} Later new details emerged about Russian tunneling.} \newline \indent In the battle for Toretsk, the Russians utilized this tactic once again. \footnote{On June 30th, 2024, the Russian news agency, \textrussian{Взгляд}, published a post on X detailing Russia's utilization of tunnels at Toretsk, stating the following: \textrussian{Крупный опорный пункт ВСУ на восточной окраине населенного пункта Кирова заняли штурмовые подразделения отряда «Ветераны» группировки войск «Центр», используя подземный туннель протяженностью более 3 км вдоль канала Северский Донец,  сообщили в Минобороны.}}. In Kirov, Sky News reported that the Russians utilized tunnels to establish control over a major Ukrainian stronghold.\footnote{["\textarabic{باستخدام نفق.. قوة روسية تسيطر على معقل أوكراني كبير.}"]}  In the months following Russia's victory at Avdiivka, the Russian military television show, \textrussian{Военная Приемка}, premiered an episode entitled, \textrussian{Связь в СВО} in July, 2024, explaining how Russian \textrussian{связисты} (i.e., soldiers specialized in communication) accompanied Russian special forces in the Avdiivka tunnels. Trained in fabricating, welding, torching, the Russian special forces combined various skills to traverse the distance from one end of the tunnel to the other. The Russians later utilized this tactic against in the Kursk direction where Russian special forces traversed one of the empty gas pipelines to dislodge the remaining Ukrainians from the region. In September, 2025, reports began to emerge from the Ukrainian military analyst, describing how Russians infiltrated deep beyond the lines of the Kupiansk fortress. Sky News Arabic subsequently picked up the story, publishing an article on September 13th, 2025.\footnote{["\textarabic{فيديو.. القوات الروسية تتسلل عبر نفق إلى كوبيانسك}"] Subsequent reports in Russia's \textrussian{Регнум} media on September 21st, 2025 appeared under the following title, "\textrussian{По трубе - в тыл противника : ВСУ готовятся к сдаче Купянска}."}\newline \indent Though a WWI tactic, Russian tunneling thus far has not followed the WWI pattern of digging large networks like the Railway Wood or the tunnels in the Messine Ridge Offensive for mining but has utilized existing pipelines for gas or other utilities to provide underground passage to the rear.

% A Russian blogging channel called, \textrussian{Крупнокалиберный Переполох}, published several episodes, examining tunnels used by Ukrainians in areas the Russians captured in 2023. The host guides viewers through the tunnels. The cart he uses for his journey is fairly common, indicating how frequently both belligerents have relied upon standard equipment to travel from one end of a tunnel to another. In one of the episodes entitled, \textrussian{В поисках оружия}, the host shows viewers weapons recovered from Ukrainian tunnels. Filmed in conjunction with the \textrussian{Донецкое Управление ФСБ}, the episode features a Spanish knife called an \emph{extrema ratio}. The hosts eats a can of \textrussian{фасолевый суп}, a canned ration from one of the Ukrainian tunnels. 

Tunneling and Tunnels 

[] - [אוניית נשק איראנית ועליה 270 טילים בדרך לעזה. 50 טילים הם טילי פאג'ר 5 משופרים, תיק דבקה, נובמבר 19, 2012]
משם סירות דייג מצריות הפועלות דרך קבע בשירות רשתות ההברחות הפלסטיניות-איראניות, תקבלנה את המטען, תפרוקנה אותו בנקודה מסוימת על חוף סיני, ומשם ינסו להעביר דרך מנהרות ההברחה לעזה. הטילים מגיעים לעזה כשהם מפורקים, וצוותי טכנאי טילים פלסטינים, איראניים, ושל חיזבאללה, ירכיבו אותם לאחר הברחתם לתוך עזה. 

[] - [איראן גייסה את הפיראטים הסומאליים כדי להבריח טילים ותחמושת לחמס, תיק דבקה, ינוי 19, 2009]
- עוד מוסרים המקורות הצבאיים שלנו כי מערכת ההברחות המסועפת הזו מבוססת על- 3 שבטים בדואיים בחצי האי סיני. המדובר בשבטי הטראבין Tarabin, היושבים בדרום ובצפון סיני, קרוב לגבולות ישראל ועזה. שבטי התייאה Tiyaha, השולטים במרכז סיני. שבטי העזזמה Azazmeh שיש להם אוכלוסיות גדולות בצפון סיני הנגב הישראלי, ירדן וסוריה
% 

}



\entry{} {} {} { 

% روسيا تتهم أوكرانيا بمهاجمة محطة لخط "ترك ستريم" للغاز


}

\entry{} {} {} {

% ["One Very Tough Russian Tank Got Hit 25 Times—And Kept Coming," \emph{Trench Art}, Oct 23, 2025]

}


\end{multicols}

%----------------------------------------------------------------------------------------
%	SECTION U
%----------------------------------------------------------------------------------------

\section*{U}

\begin{multicols}{2}

\entry{Undergrowth} {} {} {During Ukraine's 'Spring' counteroffensive, which neither happened in 'Spring' nor amounted to an offensive,' the premier British intelligence service, the cream of the crop from Oxford, Cambridge, King's College London, and Sandhurst, issued an alarming report, accounting for Ukraine's mounting difficulties in overcoming the massive undergrowth characterizing the length of the line, the great Russian general, Surovikin, constructed during the battle of Bakhmut-Artemovsk. \newline \indent Published by her majesty's Defense Intelligence on August 3rd, 2023, the intelligence update manifested three bullet points. The first stated: "[undergrowth] regrowing across the battlefields of southern Ukraine is likely one factor contributing to the generally slow progress of combat in the area." The second stated: "[the] predominately arable land in the combat zone has now been left fallow for 18 months, with the return of weeds and shrubs accelerating under the warm, damp summer conditions. The extra cover helps camouflage Russian defensive positions and makes defensive mine fields harder to clear." The third stated "Although undergrowth can also provide cover for small stealthy infantry assaults, the net effect has been to make it harder for either side to make advances." 
}


\end{multicols}


%----------------------------------------------------------------------------------------
%	SECTION W
%----------------------------------------------------------------------------------------

\section*{U}

\begin{multicols}{2}

\entry{War Mapper} {} {} {It may be that elements of the security elite attacked an account under the handle: \emph{War\_Mapper}. WM provided monthly updates on Ukraine's lost territory month over month. What is unique about WM's updates is how well programmed they were. His website provides tracking on "Overall control of Ukraine," recent "Changes to Russian held territory over last 12 months," "Control of Ukrainian Oblasts," such as Donetsk, Luhansk, Zaporizhzhia, Kharkiv, or "Control of Russian Oblasts" such as Kursk and Belgorod. WM began publishing on February 24, 2022 but abruptly stopped on April 9th, 2025. On April 9th, 2025 WM posted on X. The update described how "Russian forces advancing west of Nevske and Terny have reached and taken Katerynivka as well as positions along the Ukrainian fortification line in the area." In Zaporizhzhia, "Russia has taken the settlement of Lobkove," WM wrote, "after the recent capture of Stepove and Mali Shcherbaky. A Ukrainian counterattack into Shcherbaky has prevented its capture so far. Southwest of Pokrovsk, Russia has attacked towards, and come within 3km of, the Dnipropetrovsk Oblast boundary. In Belgorod, Russian counterattacks have pushed Popovka and Demidovka back to contested settlements. Russia crossed the Oskil again, this time at Kamianka. They now control or contest seven settlements on the west side of the Oskil. The situation in Toretsk is more clarified since the last post on the city. Though widely contested, Russian forces retained a presence in areas throughout the city. These have since come back into their control, and some forces have now moved northwest of the city." These updates were the last on the Ukraine war. Detailing the entirety of WM's contributions would require use of third party software such as TamperMonkey or ViolentMoneky to download, save or process. \newline \indent WM's charts, built on the German company, DataWrapper, facilitated insights into Russia's positive stalemate on the frontlines following Russia's receipt of Ukraine's lost initiative in the aftermath of the bloody battle of Bakhmut-Artemovsk (\emph{See} Bakhmut-Artemovsk above). It is likely that elements of the security elite attacked WM for publishing truthful information about the course of the war, a major transgression for the pro-Ukrainian supporters of the U.S-led NATO alliance, since truthful information about the course of the war inevitably points towards a Ukrainian defeat. Whatever the case may be, WM's disappearance after April 9th, 2025 indicated a major loss for the reading public's ability to analyze insights from information about the war's results quickly.}


\end{multicols}


%----------------------------------------------------------------------------------------
%	SECTION X
%----------------------------------------------------------------------------------------

\section*{X}

\begin{multicols}{2}

\entry { X } {} {} {X, the platform formerly known as X, is where the vast majority of the information from secondary commentators, commenting on information from Telegram or directly from Russia's or Ukraine's frontlines, publish most or all commentaries. While institutes usually publish directly from the institute's website, the first priority is to advertise its publication on X. The news agencies, such as the \emph{New York Times}, the \emph{Guardian}, \emph{Financial Times}, \emph{Wall Street Journal}, \emph{Foreign Policy}, \emph{Foreign Affairs}, \emph{The Economist}, or other publications, advertise publications immediately through X. It is one of the greatest sources of information on the Ukraine war. Although primary sources publish almost exclusively in Telegram, X is second to none for discovering commentary. \newline \indent It is the only place on earth where journalists, open source investigators, commentators, or analysts can openly engage in debate without reprisals. While the ruling elite from one country to the next is unified in its ambitious pursuit of control over any or all thought, X allows much to escape the clutches of what Julian Assange calls the "security elite." \newline \indent There are signs that the "security elite" has targeted specific members of the X community for expressing points of view in contradiction with the security elite's requirements for the preservation of conformed thought on the platform but X did not play an active role.
}


\end{multicols}

%----------------------------------------------------------------------------------------
%	SECTION Y
%----------------------------------------------------------------------------------------

\section*{Y}

\begin{multicols}{2}

\entry {} {} {} {} 

\end{multicols}

%----------------------------------------------------------------------------------------
%	SECTION Z
%----------------------------------------------------------------------------------------

\section*{Z}

\begin{multicols}{2}

\entry{Zaslon-M radar} {} {} {With the R-37M missile at their disposal, Russian MiG-31BMs equipped with the Zaslon-M radar have been able to fire at Ukrainian aircraft from very considerable distances,  since the beginning of the Ukraine war, without entering any of the areas in which the Ukrainian anti-aircraft defenses were operating. \newline \indent Nonetheless, the Zaslon-M radar, equipped on MiG-31BM interceptors, demonstrates how dependent the modern Russian war machine continues to be in its reliance upon the almost unrivaled treasure chest of discoveries, inventions or ideas the Soviet Union bequeathed its usurpers upon collapse. 

%https://www.deagel.com/Components/Zaslon/a003152

} 

\entry{Zhukov, Gregory} {} {} {} 

\end{multicols}

%------------------------------------------------
\end{document}